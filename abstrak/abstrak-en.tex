\begin{center}
  \large\textbf{ABSTRACT}
\end{center}

\addcontentsline{toc}{chapter}{ABSTRACT}

\vspace{2ex}

\begingroup
  \begin{center}
    \textbf{PREDIKSI KALORI YANG TERBAKAR SAAT BEROLAHRAGA DENGAN \emph{TREADMILL} BERBASIS KAMERA DAN \emph{SINGLE BOARD COMPUTER}}
  \end{center}

  % Menghilangkan padding
  \setlength{\tabcolsep}{0pt}

  \noindent
  \begin{tabularx}{\textwidth}{l >{\centering}m{3em} X}
    % Ubah kalimat berikut dengan nama mahasiswa
    Student Name / NRP  &:& Dimas Aditya Maulana Fajri / 07211940000012 \\

    % Ubah kalimat berikut dengan judul tugas akhir dalam Bahasa Inggris
    Departement         &:& Computer Engineering ELECTICS-ITS\\

    % Ubah kalimat-kalimat berikut dengan nama-nama dosen pembimbing
    Advisor             &:& Dr. Eko Mulyanto Yuniarno, S.T, M.T. \\
    %                & & 2. Wernher von Braun, S.T., M.T. \\
  \end{tabularx}
\endgroup

% Ubah paragraf berikut dengan abstrak dari tugas akhir dalam Bahasa Inggris
\emph{In this research, we proposed \lipsum[1]}

% Ubah kata-kata berikut dengan kata kunci dari tugas akhir dalam Bahasa Inggris
\emph{Keywords}: \emph{Rocket}, \emph{Anti-gravity}, \emph{Energy}, \emph{Space}.
