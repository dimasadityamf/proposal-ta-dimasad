\begin{center}
  \large\textbf{ABSTRAK}
\end{center}

\addcontentsline{toc}{chapter}{ABSTRAK}

\vspace{2ex}

\begingroup
  \begin{center}
    \textbf{PREDIKSI KALORI YANG TERBAKAR SAAT BEROLAHRAGA DENGAN \emph{TREADMILL} BERBASIS KAMERA DAN \emph{SINGLE BOARD COMPUTER}}
  \end{center}

  % Menghilangkan padding
  \setlength{\tabcolsep}{0pt}

  \noindent
  \begin{tabularx}{\textwidth}{l >{\centering}m{2em} X}
    % Ubah kalimat berikut dengan nama mahasiswa
    Nama Mahasiswa / NRP &:& Dimas Aditya Maulana Fajri / 07211940000012 \\

    % Ubah kalimat berikut dengan judul tugas akhir
    Departemen        &:&	Teknik Komputer FTEIC-ITS \\

    % Ubah kalimat-kalimat berikut dengan nama-nama dosen pembimbing
    Pembimbing        &:& Dr. Eko Mulyanto Yuniarno, S.T, M.T. \\
    %                  & & 2. Wernher von Braun, S.T., M.T. \\
  \end{tabularx}
\endgroup

\setlength{\tabcolsep}{0pt}
\textbf{Abstrak} 


% Ubah paragraf berikut dengan abstrak dari tugas akhir
Pada penelitian ini kami mengajukan \lipsum[1]

% Ubah kata-kata berikut dengan kata kunci dari tugas akhir
Kata Kunci: Roket, \emph{Anti-gravitasi}, Energi, Angkasa.
