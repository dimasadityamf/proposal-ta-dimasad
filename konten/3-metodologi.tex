\section{METODOLOGI}

% Ubah konten-konten berikut sesuai dengan isi dari metodologi

\subsection{Metode yang digunakan}

% Contoh input gambar dengan format *.jpg
\begin{figure} [ht] \centering
  % Nama dari file gambar yang diinputkan
  \includegraphics[scale=0.9]{gambar/blok diagram metodologi.png}
  % Keterangan gambar yang diinputkan
  \caption{Blok Diagram Kerja Sistem}
  % Label referensi dari gambar yang diinputkan
  \label{fig:BlokDiagram}
\end{figure}

\subsection{Bahan dan peralatan yang digunakan}

\begin{enumerate}
  \item Raspberry pi
  \item Kamera Webcam
  \item Laptop/Komputer
\end{enumerate}

\subsection{Urutan pelaksanaan penelitian}

\begin{enumerate} [label=\textbf{\arabic*}., listparindent=2em]
  \item \textbf{Akuisisi data citra}
  
  Pada tahap pertama yaitu akuisisi data citra, data diperoleh menggunakan kamera Webcam yang dimiliki oleh laptop atau kamera Webcam eksternal yang dihubungkan pada laptop ataupun komputer. Proses akuisisi data citra dilakukan dengan peraga melakukan aktivitas pada treadmill dengan ditampakkan secara jelas pada tampilan kamera Webcam. Setelah terdapat peraga dan tampak jelas pada tampilan maka data citra akan dilakukan pada tahap selanjutnya untuk dideteksi dan segmentasi pose.
  
  \item \textbf{Deteksi dan estimasi pose}
  
  Deteksi dari hasil citra untuk dapat mengetahui bentuk postur tubuh manusia menggunakan Python dengan library OpenCV yaitu MediaPipe. Metode yang digunakan pada MediaPipe menggunakan deteksi pose untuk mendeteksi postur tubuh. Segementasi dilakukan dengan cara peraga melakukan aktivitas jogging pada treadmill dengan menentukan pose melangkah.

  \item \textbf{Ekstrak fitur pose}
  
  Fitur dibuat berdasarkan segmentasi pose yang telah ditentukan dan dilakukan deteksi. Semua fitur dipersiapkan sebagai kombinasi dataset yang nantinya akan digunakan pada training. Setelah menentukan fitur yang akan diekstrak, dilakukan ekstrak fitur untuk mendapatkan setiap data yang dibutuhkan dengan setiap percobaan dari kombinasi segmentasi pose. Hasil yang didapat dari ekstrak fitur berupa data set yang nantinya akan dilakukan training untuk model yang diinginkan.

  \item \textbf{Klasifikasi}
  
  Fitur yang telah dilakukan ekstraksi maka kemudian dilakukan training untuk memperoleh model deteksi. Model deteksi dari data set akan digunakan untuk melatih model dari sebuah algoritma pada Machine Learning. Dalam melakukan klasifikasi menggunakan LSTM (Long Short-Term Memory) Neural Networks. Proses training ini bertujuan agar nantinya komputasi yang dilakukan dalam proses deteksi akan dapat diolah berdasarkan akuisisi data citra menjadi bentuk atau pola pemahaman yang diinginkan. 

  \item \textbf{Hasil deteksi}
  
  Setelah dilakukan training dan klasifikasi, akan didapat model deteksi yang diinginkan. Bentuk hasil klasifikasi yang dibuat adalah mendeteksi pose aktivitas dengan dapat menghitung langkah dan waktu yang ditempuh. Nilai langkah dan waktu yang ditempuh akan digunakan dalam perhitungan selanjutnya.

  \begin{enumerate}[label=\textbf{\alph*}., listparindent=2em]
    \item Hasil deteksi langkah
    
    Banyaknya jumlah langkah yang didapat saat hasil deteksi digunakan sebagai nilai variable pertama yang akan digunakan dalam penentuan perhitungan kalori. Langkah dideteksi dan dihitung seberapa banyak langkah yang dilakukan saat proses deteksi. Nilai banyaknya jumlah langkah akan disimpan dan akan digunakan pada saat proses perhitungan kalori setelah proses deteksi telah selesai dilakukan.

    \item Hasil perhitungan waktu tempuh
    
    Waktu tempuh saat proses deteksi merupakan nilai variabel kedua yang akan digunakan dalam penentuan perhitungan kalori. Waktu tempuh dimulai saat dideteksi pertama kali nilai langkah yang ditemukan hingga saat akhir langkah tidak ada penambahan kembali yang menandakan proses deteksi telah selesai. Nilai waktu akan dibutuhkan dalam satuan waktu menit untuk proses perhitungan kalori.

  \end{enumerate}

  \item \textbf{Estimasi kalori}
  
  Perhitungan kalori dilakukan dengan mengacu pada nilai satuan ukuran MET (Metabolic Equivalent). Satuan MET akan mendapat pengukuran untuk konsumsi oksigen dan pembakaran kalori. Pada perhitungan ini akan difokuskan pada pembakaran kalori dan perhitungannya untuk mendapatkan hasil perhitungan pembakaran kalori total saat melakukan suatu aktivitas. Nilai dari satuan MET dapat didefinisikan pada Persamaan 3.1.

  %Persamaan
  
  Berdasarkan nilai satuan ukuran MET, didapatkan suatu persamaan untuk menghitung pembakaran kalori yang didefiniskan pada Persamaan 3.2.

  %Persamaan

  Pada persamaan pembakaran kalori yang akan digunakan untuk melakukan perhitungan pembakaran kalori dari aktivitas yang dilakukan dibutuhkan beberapa nilai variabel untuk mendapatkah hasil total pembakaran kalori. Nilai-nilai yang dibutukan merupakan nilai dari jenis aktivitas yang akan mempengaruhi nilai MET, nilai berat badan (BB) dari seseorang yang melakukan aktivitas, dan waktu yang dibutuhkan dalam melakukan aktivitas tersebut.

  \begin{enumerate}[label=\textbf{\alph*}., listparindent=2em]
    \item MET (Metabolic Equivalent)
    
    Nilai MET merupakan nilai yang sangat penting pada persamaan ini. Setiap aktivitas memiliki nilai MET yang berbeda-beda dan telah ditentukan oleh peneliti yang telah merangkum banyak aktivitas untuk ditentukan berapa nilai MET yang dihasilkan. Pada aktivitas olahraga yang difokuskan saat ini adalah jogging pada treadmill juga memiliki perbedaan nilai MET yang dipengaruhi oleh kecepatan jogging. Dengan begitu nilai kecepatan aktivitas jogging perlu diketahui untuk nantinya akan digunakan menentukan nilai MET yang dihasilkan.

    Kecepatan aktivitas jogging dapat diketahui dengan nilai hasil deteksi yang telah dilakukan. Nilai banyaknya langkah dan waktu tempuh digunakan untuk menentukan kecepatan. Pada persamaan umum dalam menentukan kecepatan, diketahui jika persamaan untuk menentukan kecepatan dapat didefinisikan pada Persamaan 3.3.

    %Persamaan

    Dengan didapatnya nilai banyaknya langkah yang didapat dari hasil deteksi, maka untuk menentukan jarak yang ditempuh dapat dengan mengalikan banyaknya langkah dengan panjang rata-rata langkah kaki dengan nilai 0,76 m. Sehingga untuk menentukan jarak dapat didefinisikan pada Persamaan 3.4.

    %Persamaan

    Dengan begitu, didapat nilai kecepatan dengan membagi jarak yang ditempuh berdasarkan banyaknya langkah dengan waktu tempuh yang telah didapat dari hasil deteksi.

    \item Berat badan
    
    Berat badan (BB) juga merupakan nilai yang dibutuhkan dalam melakukan perhitungan kalori yang terbakar karena termasuk dalam persamaan yang digunakan. Nilai berat badan didapat dengan cara memberikan input tambahan sebelum melakukan aktivitas untuk menghitung pembakaran kalori sesuai dengan berat badan dari seseorang yang melakukan aktivitas tersebut. Jika tidak adanya nilai berat badan yang dimasukkan ke dalam perhitungan sesuai dengan seseorang yang melakukan aktivitas, dapat diberikan nilai rata-rata berat badan orang dewasa yaitu 70 kg. Sehingga dapat dilakukan proses perhitungan pembakaran kalori yang sesuai.

    \item Waktu tempuh
  \end{enumerate}


  \item \textbf{Implementasi pada perangkat Singel Board Computer}
  
  Implementasi dilakukan sebagai testing dengan akuisisi citra secara realtime menggunakan model training machine learning yang telah dibuat. Setelah proses telah berjalan dengan baik pada testing menggunakan laptop/komputer, persiapan perangkat Single Board Computer untuk dapat menampung segala kebutuhan dalam mendeteksi dan memprediksi hasil kalori yang diinginkan. Implementasi juga dapat ditambahkan dengan memberikan hasil visualisasi menggunakan user interface sebagai tampilan yang dapat divisualkan untuk mempermudah pembacaan hasil.


\end{enumerate}

\subsubsection{Jadwal Kegiatan}

% Ubah tabel berikut sesuai dengan isi dari rencana kerja
\newcommand{\w}{}
\newcommand{\G}{\cellcolor{gray}}
\begin{table}[h!]
  \begin{tabular}{|p{3.5cm}|c|c|c|c|c|c|c|c|c|c|c|c|c|c|c|c|}

    \hline
    \multirow{2}{*}{Kegiatan} & \multicolumn{16}{|c|}{Minggu} \\
    \cline{2-17} &
    1 & 2 & 3 & 4 & 5 & 6 & 7 & 8 & 9 & 10 & 11 & 12 & 13 & 14 & 15 & 16 \\
    \hline

    % Gunakan \G untuk mengisi sel dan \w untuk mengosongkan sel
    Studi pustaka &
    \G & \G & \G & \G & \w & \w & \w & \w & \w & \w & \w & \w & \w & \w & \w & \w \\
    \hline

    Akuisisi data citra &
    \w & \w & \G & \G & \G & \w & \w & \w & \w & \w & \w & \w & \w & \w & \w & \w \\
    \hline

    Deteksi dan segmentasi pose &
    \w & \w & \w & \G & \G & \G & \w & \w & \w & \w & \w & \w & \w & \w & \w & \w \\
    \hline

    Ekstrak fitur &
    \w & \w & \w & \w & \G & \G & \G & \w & \w & \w & \w & \w & \w & \w & \w & \w \\
    \hline

    Training dataset &
    \w & \w & \w & \w & \G & \G & \G & \G & \w & \w & \w & \w & \w & \w & \w & \w \\
    \hline

    Estimasi kalori &
    \w & \w & \w & \w & \w & \w & \G & \G & \G & \G & \G & \G & \w & \w & \w & \w \\
    \hline

    Implementasi pada perangkat single board computer &
    \w & \w & \w & \w & \w & \w & \w & \w & \w & \w & \w & \G & \G & \G & \w & \w \\
    \hline

    Uji coba dan evaluasi &
    \w & \w & \w & \w & \w & \w & \G & \G & \G & \G & \G & \G & \G & \G & \G & \w \\
    \hline

    Penyusunan laporan &
    \G & \G & \G & \G & \G & \G & \G & \G & \G & \G & \G & \G & \G & \G & \G & \G \\
    \hline

  \end{tabular}
\end{table}
