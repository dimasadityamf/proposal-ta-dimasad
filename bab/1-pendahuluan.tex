\chapter{PENDAHULUAN}
\label{chap:pendahuluan}

% Ubah bagian-bagian berikut dengan isi dari pendahuluan

Penelitian ini dilatarbelakangi oleh \lipsum[1][1-5]

\section{Latar Belakang}
\label{sec:latarbelakang}

Obesitas merupakan keadaan dimana terdapat penumpukan lemak pada tubuh seseorang yang menyebabkan berat badan berada pada nilai di atas normal. Indikasi yang dapat digunakan untuk menilai jika seseorang menderita obesitas berdasarkan nilai body mass index (BMI) yang lebih dari 30 kg/m2. Obesistas disebabkan oleh kalori yang dikonsumsi tidak seimbang dengan kalori yang digunakan oleh tubuh. Salah satu hal yang dapat digunakan untuk mencegah obesitas dan mengurangi kelebihan berat badan dengan melakukan olahraga. 

Olahraga merupakan suatu bentuk aktivitas fisik dalam kegiatan jasmani yang dilakukan secara terstruktur dengan melibatkan pergerakan tubuh secara berulang-ulang. Aktvitas olahraga dilakukan dengan tujuan untuk memelihara kesehatan dan memperkuat otot-otot tubuh. Olahraga menjadi kegiatan yang sangat dekat dengan aktivitas manusia sebagai salah satu kebutuhan hidup dalam memberikan manfaat berupa kesehatan dan kebugaran tubuh.

Aktivitas olahraga dinilai bermanfaat dan sesuai prosedur dengan melihat bagaimana kualitas aktivitas olahraga yang telah dilakukan. Kualitas aktivitas olahraga dapat diukur berdasarkan jumlah energi yang dikeluarkan selama melakukan aktivitas olahraga. Energi yang dikeluarkan akan membantu meningkatkan jumlah pembakaran kalori pada tubuh. Jumlah energi yang dikeluarkan selama melakukan aktivitas olahraga akan berbeda-beda tergantung dari jenis aktivitas, durasi dan beberapa faktor pada individu.

\section{Rumusan Masalah}
\label{sec:permasalahan}

Aktivitas yang dilakukan pada treadmill dengan perhitungan pembakaran kalori menggunakan perhitungan manual dengan menggunakan beberapa faktor individu masih belum cukup efektif. karena kegiatan olahraga yang dilakukan masih belum terukur secara detail dari bagaimana aktivitas tersebut dilakukan. Gerak tubuh dan postur yang berbeda akan menghasilkan jumlah aktivitas dan pembakaran kalori yang berbeda pula. Oleh karena itu, diperlukan sistem untuk dapat melakukan perhitungan pembakaran kalori yang lebih praktis dan akurat untuk berolahraga pada treadmill.

\section{Batasan Masalah}
\label{sec:batasanmasalah}

Tujuan dari \lipsum[1][1-3] adalah:

\begin{enumerate}[nolistsep]

  \item Membuat \lipsum[2][1-3]

  \item \lipsum[3][1-3]

\end{enumerate}

\section{Tujuan}
\label{sec:Tujuan}

Batasan-batasan dari \lipsum[1][1-3] adalah:

\begin{enumerate}[nolistsep]

  \item Mempermudah \lipsum[2][1-3]

  \item \lipsum[3][1-5]

  \item \lipsum[4][1-5]

\end{enumerate}

\section{Sistematika Penulisan}
\label{sec:sistematikapenulisan}

Laporan penelitian tugas akhir ini terbagi menjadi \lipsum[1][1-3] yaitu:

\begin{enumerate}[nolistsep]

  \item \textbf{BAB I Pendahuluan}

  Bab ini berisi \lipsum[2][1-5]

  \vspace{2ex}

  \item \textbf{BAB II Tinjauan Pustaka}

  Bab ini berisi \lipsum[3][1-5]

  \vspace{2ex}

  \item \textbf{BAB III Desain dan Implementasi Sistem}

  Bab ini berisi \lipsum[4][1-5]

  \vspace{2ex}

  \item \textbf{BAB IV Pengujian dan Analisa}

  Bab ini berisi \lipsum[5][1-5]

  \vspace{2ex}

  \item \textbf{BAB V Penutup}

  Bab ini berisi \lipsum[6][1-5]

\end{enumerate}
